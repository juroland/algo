\documentclass[handout]{beamer}
\usepackage[utf8]{inputenc}
\usetheme{metropolis}           % Use metropolis theme
\title{Algorithmique Avancée}
\subtitle{Fondamentaux}
\author{Julien Roland}

\let\emph\relax % there's no \RedeclareTextFontCommand
\DeclareTextFontCommand{\emph}{\bfseries\em}

\usepackage{amsmath}
\newcommand{\Op}[2]{\texttt{\underline{#1}#2}}

\newcommand{\N}{\mathbb{N}}
\newcommand{\R}{\mathbb{R}}

\usepackage{listings}
\usepackage{algorithmicx}
\usepackage{algpseudocode}

\begin{document}
\lstset{language=c++, basicstyle=\small, columns=fullflexible}
    
\setbeamercovered{dynamic}
    
\maketitle

\begin{frame}[t]{Références}
    
Références triées par ordre d'importance:

\begin{enumerate}
    \item \emph{The Algorithm Design Manual} par Steven Skiena.
    \item \emph{Introduction to Algorithms} par Thomas H. Cormen.
    \item \emph{Data Structures and Algorithms in C++} par Goodrich, Tamassia, et Mount.
\end{enumerate}
\end{frame}

\begin{frame}{Contenu du cours}
\begin{itemize}
    \item Fondamentaux
    \item Implémentation
    \item Structures de données
    \item Les graphes
    \item Les algorithmes gloutons
    \item Programmation dynamique
\end{itemize}
\end{frame}

\begin{frame}[t]{Algorithme}
    \begin{flushright}
        \textit{"Un ensemble fini de règles qui donne une séquence d'opérations pour résoudre un problème particulier."}\\
        Donald Knuth, The Art of Computer Programming
    \end{flushright}
    Caractéristiques importantes pour tous les algorithmes :
    \begin{itemize}
        \item Fini
        \item Défini (précis)
        \item Entrée(s)
        \item Sortie(s)
        \item \textit{"Effective"}
    \end{itemize}
\end{frame}

\begin{frame}[t]{Structures de données}
    \begin{flushright}
        \textit{"Une structure de données est un moyen de stocker et organiser des données afin d'en faciliter l'accès et leur modification."}\\
        Cormen et al. Introduction to Algorithms
    \end{flushright}

    \vfill

    \begin{flushright}
        \textit{"Une structure de données est une manière systématique permettant d'organiser et d'accéder à des données"}\\
        Goodrich et al. Data Structures and Algorithms in C++
    \end{flushright}
    
    \vfill

    Quelques exemples ?
    
\end{frame}

\begin{frame}[t]{Types de données abstraits}
    \begin{flushright}
        \textit{"Un type de donnée abstrait définit un ensemble d'objets abstraits caractérisés par les opérations qui leurs sont applicables"}\\
        B. Liskov et S. Zilles, Programming with abstract data types.
    \end{flushright}

    Parmi les exemples suivants, identifier les structures de données et les ADTs :

    \begin{enumerate}
        \item Liste
        \item File (queue)
        \item Liste simplement chaînée
        \item Tableau
        \item Pile (stack)
        \item Dictionnaire
        \item Arbre
    \end{enumerate}
\end{frame}

\begin{frame}[t]{ADT Liste}
    Opérations généralement définies pour une liste L :
    \begin{itemize}
        \item \lstinline{begin(L)}
        \item \lstinline{end(L)}
        \item \lstinline{push_front(L, e)}
        \item \lstinline{push_back(L, e)}
        \item \lstinline{insert(L, p, e)}
        \item \lstinline{pop_front(L)}
        \item \lstinline{pop_back(L)}
        \item \lstinline{erase(L, p)}
    \end{itemize}
    Comment implémenter un tel ADT à l'aide d'un Tableau ?
\end{frame}

\begin{frame}[t]{ADT Stack}
    Opérations définies pour un stack S:
    \begin{itemize}
        \item push(S, x)
        \item pop(S)
    \end{itemize}

    Quelles implémentations ?
\end{frame}

\begin{frame}[fragile]{ADT en C++}
    En C++, un type de données abstrait est défini par une classe sans données membres publiques.
    \begin{lstlisting}
        template<typename T>
        class Stack {
        public:
            void push(const T& x);
            void pop();
            const T& top();
            int size() const;
            bool empty() const;
        };
    \end{lstlisting}
    Dans la STL, \lstinline{std::stack} est un adaptateur de conteneurs :
    \begin{lstlisting}
    template<class T, class Container = std::deque<T>> class stack;
    \end{lstlisting}
\end{frame}

\begin{frame}[t]{Pseudo-code}
    Pourquoi utiliser un pseudo-code au lieu d'un langage de programmation ?
    \begin{itemize}
        \item Permet d'utiliser ce qui est le plus précis, expressif et concis.
        \item Combiner code et phrases en français. 
    \end{itemize}
    Il n'y a pas de définition précise de pseudo-code !

    \begin{flushright}
        \textit{Pseudocode is perhaps the most mysterious of the bunch, but it is best defined as a
        programming language that never complains about syntax errors.}\\
        Skiena, The Algorithm Design Manual
    \end{flushright}
\end{frame}

\begin{frame}[fragile]{Pseudo-code}
Il doit être suffisamment \emph{clair} pour être en mesure de le le mettre en oeuvre dans un language de programmation \emph{sans ambiguïté}.

\begin{algorithmic}
    \Function{InsertionSort}{$A$}
    \Comment{A, un tableau indicé de 0 à n-1}
    \ForAll{$i \leftarrow 1$ \textbf{to} $n - 1$}
        \State $key \leftarrow A[i]$
        \State $j \leftarrow i - 1$\\
        \Comment{Insertion de key dans la séquence triée A[0...i-1]}
        \While{$j \geqslant 0$ \textbf{and} $A[j] > key$}
            \State $A[j + 1] \leftarrow A[j]$
            \State $j \leftarrow j - 1$
        \EndWhile 
        \State $A[j + 1] \leftarrow key$
    \EndFor
    \EndFunction
\end{algorithmic}

\end{frame}

\begin{frame}[fragile]{Pseudo-code}
Un pseudo-code devrait être uniquement composé d'opérations primitives et de structures de contrôles.\\
\vfill
Exemple à ne pas suivre :
\vfill
\begin{algorithmic}
    \Function{InsertionSort}{$A$}
    \ForAll{$i \leftarrow 1$ \textbf{to} $n - 1$}
        \State Insertion de $A[i]$ dans la séquence triée $A[0...i-1]$
    \EndFor
    \EndFunction
\end{algorithmic}

Le temps d'exécution de chaque étape doit être en temps constant (c-à-d, ne pas dépendre de la taille de l'instance du problème).
\end{frame}

\section{Analyse des algorithmes}

\begin{frame}[t]{Modèle de calcul RAM}
    Nous considérons un modèle d'ordinateur hypothétique appelé \emph{Random Access Machine} où chaque \emph{opération primitive} nécessite exactement \emph{une étape} de temps :
    \begin{itemize}
        \item Opération arithmétique et logique : +, *, ==,...
        \item Affecter une valeur à une variable
        \item Appeler une fonction
        \item Accès mémoire: par exemple indexer un tableau
        \item Suivre une référence vers un objet
        \item ...
    \end{itemize}
    Ainsi, nous mesurons le temps d'exécution en comptant le nombre d'étapes.
\end{frame}

\begin{frame}[fragile]{Modèle de calcul RAM}
    \begin{lstlisting}
        int min(int, int);

        int min1(int v[], int n) {
            int m = v[0]
            for (int i = 0; i < n; n++)
                m = min(v[i], m);
            return m;
        }

        int min2(int v[], int n) {
            int m = v[0];
            for (int i = 0; i < n; i += 2) 
                m = min(v[i], m);
            for (int i = 1; i < n; i += 2) 
                m = min(v[i], m);
            return m;
        }
    \end{lstlisting}
\end{frame}

\begin{frame}[fragile]{Modèle de calcul RAM}
    Discussion:
    \begin{itemize}
        \item Ce modèle est-il réalise ?
        \item Quels sont les limites ?
        \item En quoi est-il utile en pratique ?
    \end{itemize}
\end{frame}

\begin{frame}{Pire cas, Meilleur cas, Cas moyen}
    \centering
    \includegraphics[scale=0.5]{figure/best_worst_average.pdf} 
\end{frame}

\begin{frame}{Rappels}
\begin{itemize}
    \item Espérance d'une variable aléatoire discrète $X$ :
    $$
    E[X] = \sum_{i = 1}^n x_i p_i
    $$
    \item Coefficient binomial :
    $$
    {n \choose k} = C_n^k\, = \frac{n!}{k!(n-k)!}
    $$
    \item Le $n$-ième nombre triangulaire :
    $$
    \sum_{k=1}^n k = \frac{n(n+1)}{2}
    $$
\end{itemize}
\end{frame}

\begin{frame}[fragile]{Pire cas, Meilleur cas, Cas moyen}
    \begin{lstlisting}
        int index(const char *s, char ch)
        {
            for (const char *p = s ;; ++p) {
                if (*p == ch)
                    return p - s;
                if (*p == '\0')
                    return -1;
            }
        }
    \end{lstlisting}
    Quels sont les pire cas, meilleur cas et cas moyen ?
\end{frame}


\begin{frame}[t]{Majorant asymptotique}
    \begin{itemize}
        \item $f : \N \rightarrow \R$ et $g : \N \rightarrow \R$
        \item $O(g(n)) = \big\{ f(n) :$ il existe des constantes positives $c$ et $n_0$ telles que $0 \leqslant f(n) \leqslant c g(n)$ pour tout $n \geqslant n_0\big\}$
    \end{itemize}
    \begin{center}
        \includegraphics[scale=0.4]{figure/upper_bound.pdf} 
    \end{center}
    Il est généralement admis de noter, par exemple, $3n^2 - 100m + 6 = O(n^2)$.
\end{frame}

\begin{frame}[t]{Classes de complexités}
    La notation en "grand O" permet d'évaluer plus simplement le temps d'exécution d'un algorithme comme "taux d'accroissement" par rapport à la taille du problème.

    \begin{itemize}
        \item $O(1)$ : constant
        \item $O(\log n)$ : logarithmique
        \item $O(n)$ : linéaire
        \item $O(n \log n)$ : quasi linéaire
        \item $O(n^2)$ : quadratique
        \item $O(n^3)$ : cubique
        \item $O(c^n)$ : exponentielle (avec une constante $c > 1$)
        \item $O(n!)$ : factorielle
    \end{itemize}
\end{frame}

\begin{frame}[t]{Exercices}
À l'aide de la définition de la notation en \emph{grand O}, démontrer les résultats suivants :
\begin{enumerate}
    \item $f(n) = 8n + 2$ est en $O(n)$ (également noté $8n + 2 = O(n)$)
    \item $n^2 = O(n^3)$
    \item $n^3 + n^2 + n + 1 = O(n^3)$
\end{enumerate}
\end{frame}

\begin{frame}[t]{Notations et Résultats}
\begin{itemize}
    \item Relation de dominance :
        $$
        f(n) = O(g(n))\text{, alors}\ g \gg f
        $$
    \item La somme de deux fonctions :
        $$
        O(f(n)) + O(g(n)) \rightarrow O(max(f(n), g(n))
        $$
    \item Transitivité :
        $$
        f(n) = O(g(n))\ \text{et}\ g(n) = O(h(n)) \rightarrow f(n) = O(h(n))
        $$
\end{itemize}
\end{frame}

\begin{frame}[t]{Analyse de complexité}
Deux algorithmes A et B permettent de résoudres toutes instances d'un même problème. Les temps d'exécutions sont respectivements en $O(n)$ et en $O(n^2)$
\vfill
Lesquelles des affirmations suivantes sont vraies :
\begin{enumerate}
\item L'algorithme A est dit "asymptotiquement meilleur" que B.
\item Pour toutes instances, le temps de cacul de A est inférieur au temps de calcul de B.
\item Il est potentiellement possible de trouver certaines instances pour lesquelles le temps de calculs de A est deux fois suppérieur au temps de calcul de B.
\item Le choix de l'algorithme A (au détriment de l'algorithme B) permet d'anticiper un passage à l'échelle (faire face à une montée en charge).
\end{enumerate}
\end{frame}


\end{document}
    