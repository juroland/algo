\documentclass{beamer}
\usepackage[utf8]{inputenc}
\usetheme{metropolis}           % Use metropolis theme
\title{Algorithmique Avancée}
\subtitle{ADT : Ordered Map}
\author{Julien Roland}

\let\emph\relax % there's no \RedeclareTextFontCommand
\DeclareTextFontCommand{\emph}{\bfseries\em}

\newcommand{\Op}[2]{\texttt{\underline{#1}#2}}

\usepackage{listings}


\begin{document}
\lstset{language=c++, basicstyle=\small, columns=fullflexible}

% \setbeamercovered{dynamic}
\setbeamercovered{invisible}

\maketitle


\begin{frame}{ADT : Map}
    Outre les opérations de l'ADT Map, les opérations suivantes sont disponibles :
\begin{block}{Opérations (pour une Ordered Map M)}
    \begin{description}[\Op{predecessor}{(M, x)} :]
        \item[\Op{min}{(M)} : ] retourne un pointeur vers l'entrée de M avec la plus petite valeur
            de clé. 
        \item[\Op{max}{(M)} : ] retourne un pointeur vers l'entrée de M avec la plus grande valeur
            de clé.
        \item[\Op{successor}{(M, x)} :] retourne un pointeur vers l'entrée de M suivant l'entrée
            pointée par x. Si x pointe vers la dernière entrée de M, alors retourne ``NULL".
        \item[\Op{predecessor}{(M, x)}]
    \end{description}
    Dans tous les cas, si M est vide, alors retourne ``NULL".
\end{block}
\end{frame}

\begin{frame}[fragile]{\texttt{std::unordered\_map}}
    \begin{lstlisting}
    // from http://en.cppreference.com/w/cpp/container/map/begin
    #include <iostream>
    #include <map>

    int main() {
        std::map<int, float> num_map;
        num_map[4] = 4.13;
        num_map[9] = 9.24;
        num_map[1] = 1.09;

        for (auto it = num_map.begin(); it != num_map.end(); ++it) {
            std::cout << it->first << ", " << it->second << '\n';
        }
    }
    \end{lstlisting}
\end{frame}

\begin{frame}{Implémentations}
    \begin{block}{Structures de données}
        \begin{itemize}
            \item Liste (doublement ou simplement chaînée) triée
            \item Tableau trié
            \item Arbre binaire de recherche (Binary Search Tree - BST)
            \item ...
        \end{itemize}
    \end{block}
\end{frame}

\begin{frame}[t]{Arbre Binaire de Recherche}
Un arbre binaire T tel que chaque nœud $x$ contient une paire $(key, value)$ et satisfait la
propriété suivante:
\begin{itemize}
    \item Si $y$ est un nœud dans le sous arbre gauche de $x$, alors $y.key \leqslant x.key$.
    \item Si $y$ est un nœud dans le sous arbre droit de $x$, alors $y.key \geqslant x.key$.
\end{itemize}
\end{frame}

\begin{frame}[fragile]{Parcours infixe}
La propriété de l'arbre de recherche nous permet de lister l'ensemble des entrées de l'arbre de
manière triée à l'aide d'un algorithme récursif.
\pause
\begin{lstlisting}
        inorder_tree_walk(x) {
            if x == NULL {
                return
            }
            inorder_tree_walk(x.left)
            print(x.key)
            inorder_tree_walk(x.right)
        }
\end{lstlisting}
\end{frame}

\begin{frame}[fragile]{Parcours infixe : complexité}
    Quelle est la complexité de cet algorithme ?\\
    \pause
    \vfill
    Si $x$ est la racine d'un arbre composé de $n$ nœuds, alors la complexité de
    \texttt{inorder\_tree\_walk(x)} est en $O(n)$. (pour chaque nœud, la fonction est appelée 2 fois)
\end{frame}


\begin{frame}[fragile]{Recherche}
    \pause
    \begin{lstlisting}
        tree_find(x, k) {
            if x == NULL or k == x.key {
                return x
            }
            if k < x.key {
                return tree_find(x.left, k)
            }
            return tree_find(x.right, k)
        }
    \end{lstlisting}

    Quelle est la complexité de cet algorithme ?
\end{frame}

\begin{frame}[fragile]{Successeur}
    \pause
    \begin{lstlisting}
        tree_successor(x) {
            if x.right != NULL
                return tree_minimum(x.right)
            y = x.parent
            while y != NULL and x == y.right {
                x = y
                y = x.parent
            }
            return y
        }
    \end{lstlisting}
\end{frame}

\lstset{language=c++, basicstyle=\scriptsize, columns=fullflexible}

\begin{frame}[fragile]{Insertion}
    Objectif : maintenir la propriété de l'arbre binaire de recherche.
    \pause
    \begin{lstlisting}
        // un noeud v tel que v.left = NULL et v.right = NULL
        tree_insert(T, v) {
            y = NULL; x = T.root
            while x != NULL {
                y = x
                if v.key < x.key:
                    x = x.left
                else:
                    x = x.right
            }
            v.parent = y
            if y == NULL:   // Empty Tree T
                T.root = v
            else if v.key < y.key:
                y.left = v
            else:
                y.right = v
        }
    \end{lstlisting}
\end{frame}

\begin{frame}[t]{Exercice}
    \begin{enumerate}
        \item Pour l'ensemble $\{1, 4, 5, 10, 16, 17, 21\}$ de clés, dessiner un arbre binaire de recherche de hauteur 2 et 6.
        \item Donner une version non-récursive de l'algorithme inorder\_tree\_walk. (avec et sans l'utilisation d'un stack).
        \item Donner une version non récursive (itérative) des fonctions tree\_find(), tree\_min(), et tree\_max().
        \item Écrire la fonction tree\_predecessor(x).
        \item Écrire les version récursives des procédures tree\_min() et tree\_max().
    \end{enumerate}
\end{frame}
\end{document}
